%%%%%%%%%%%%%%%%%
% This is an sample CV template created using altacv.cls
% (v1.3, 10 May 2020) written by LianTze Lim (liantze@gmail.com). Now compiles with pdfLaTeX, XeLaTeX and LuaLaTeX.
% (v1.7.1b, 11 Jan 2024) forked by Nicolás Omar González Passerino (nicolas.passerino@gmail.com)
%
%% It may be distributed and/or modified under the
%% conditions of the LaTeX Project Public License, either version 1.3
%% of this license or (at your option) any later version.
%% The latest version of this license is in
%%    http://www.latex-project.org/lppl.txt
%% and version 1.3 or later is part of all distributions of LaTeX
%% version 2003/12/01 or later.
%%%%%%%%%%%%%%%%

%% If you need to pass whatever options to xcolor
\PassOptionsToPackage{dvipsnames}{xcolor}

%% If you are using \orcid or academicons
%% icons, make sure you have the academicons
%% option here, and compile with XeLaTeX
%% or LuaLaTeX.
% \documentclass[10pt,a4paper,academicons]{altacv}

\documentclass[10pt,a4paper,ragged2e,withhyper]{altacv}

\usepackage{academicons}
%% AltaCV uses the fontawesome5 and academicons fonts
%% and packages.
%% See http://texdoc.net/pkg/fontawesome5 and http://texdoc.net/pkg/academicons for full list of symbols. You MUST compile with XeLaTeX or LuaLaTeX if you want to use academicons.

%% Fork v1.6.5c: Overwriting sloppy environment to ignore any spaces and be used to solve overlapping cvtags
\newenvironment{sloppypar*}{\sloppy\ignorespaces}{\par}

% Change the page layout if you need to
\geometry{left=1.2cm,right=1.2cm,top=1cm,bottom=1cm,columnsep=0.75cm}

% The paracol package lets you typeset columns of text in parallel
\usepackage{paracol}

% Change the font if you want to, depending on whether
% you're using pdflatex or xelatex/lualatex
\ifxetexorluatex
  % If using xelatex or lualatex:
  \setmainfont{Roboto Slab}
  \setsansfont{Lato}
  \renewcommand{\familydefault}{\sfdefault}
\else
  % If using pdflatex:
  \usepackage[rm]{roboto}
  \usepackage[defaultsans]{lato}
  % \usepackage{sourcesanspro}
  \renewcommand{\familydefault}{\sfdefault}
\fi

\definecolor{PrimaryColor}{HTML}{383838}
\definecolor{SecondaryColor}{HTML}{9D0137}
\definecolor{ThirdColor}{HTML}{DCEA8E}
\definecolor{BodyColor}{HTML}{666666}
\definecolor{EmphasisColor}{HTML}{2E2E2E}
\definecolor{BackgroundColor}{HTML}{FFFFFF}

\colorlet{name}{PrimaryColor}
\colorlet{tagline}{SecondaryColor}
\colorlet{heading}{PrimaryColor}
\colorlet{headingrule}{ThirdColor}
\colorlet{subheading}{SecondaryColor}
\colorlet{accent}{SecondaryColor}
\colorlet{emphasis}{EmphasisColor}
\colorlet{body}{BodyColor}
\pagecolor{BackgroundColor}

% Change some fonts, if necessary
\renewcommand{\namefont}{\Huge\rmfamily\bfseries}
\renewcommand{\personalinfofont}{\small\bfseries}
\renewcommand{\cvsectionfont}{\LARGE\rmfamily\bfseries}
\renewcommand{\cvsubsectionfont}{\large\bfseries}

% Change the bullets for itemize and rating marker
% for \cvskill if you want to
\renewcommand{\itemmarker}{{\small\textbullet}}
\renewcommand{\ratingmarker}{\faCircle}

%% sample.bib contains your publications
%% \addbibresource{main.bib}

\begin{document}
    \name{John Doe}
    \tagline{Test Developer}

    \personalinfo{%
    \begin{minipage}{0.6\textwidth}
      \email{john_doe@email.com}\\[0.2cm]
      \phone{+01-2345-678901}\\[0.2cm]
      \orcid{0000-0000-0000-0000}
    \end{minipage}%
    \begin{minipage}{0.4\textwidth}
      \github{githubUser}\\[0.2cm]
      \linkedin{linkedinUser}\\[0.2cm]
      \scholar{scholarID}
    \end{minipage}%
    }
    
    \makecvheader
    %% Depending on your tastes, you may want to make fonts of itemize environments slightly smaller
    % \AtBeginEnvironment{itemize}{\small}
    
    %% Set the left/right column width ratio to 6:4.
    \columnratio{0.25}

    % Start a 2-column paracol. Both the left and right columns will automatically
    % break across pages if things get too long.
    \begin{paracol}{2}
        % ----- TECH STACK -----
        \cvsection{TECH LANGS}
            %% Fork v1.6.5c: The sloppypar* environment is used to avoid tags overlapping with section width
            \begin{sloppypar*}
                %% Fork 1.7.1b: Now in case you want to highlight any tag, just add a '/true' property next to its text and it will change tag's text and border colors.
                \cvtags{One/true, Two, Three/true, Four, Five/true, Six, Seven/true, Eight, Nine/true}
            \end{sloppypar*}
        % ----- TECH STACK -----

        % ----- TECH STACK -----
        \cvsection{TECH STACK}
            %% Fork v1.6.5c: The sloppypar* environment is used to avoid tags overlapping with section width
            \begin{sloppypar*}
                %% Fork 1.7.1b: Now in case you want to highlight any tag, just add a '/true' property next to its text and it will change tag's text and border colors.
                \cvtags{One/true, Two, Three/true, Four, Five/true, Six, Seven/true, Eight, Nine/true}
            \end{sloppypar*}
        % ----- TECH STACK -----
        
        % ----- DATA STACK -----
        \cvsection{DATA STACK}
            \begin{sloppypar*}
                \cvtags{Uno, Dos/true, Tres, Cuatro/true, Cinco, Seis/true, Siete, Ocho/true, Nueve}
            \end{sloppypar*}
        % ----- DATA STACK -----

        % ----- OTHERS -----
        \cvsection{OTHERS}
            \begin{sloppypar*}
                \cvtags{Uno, Dos/true, Tres, Cuatro/true, Cinco, Seis/true, Siete, Ocho/true, Nueve}
            \end{sloppypar*}
        % ----- OTHERS -----
        
        % ----- LANGUAGES -----
        \cvsection{Languages}
            \cvlang{Lang 1}{Native}\\
            \divider

            \cvlang{Lang 2}{Basic / A2}
            \medskip
            %% Yeah I didn't spend too much time making all the
            %% spacing consistent... sorry. Use \smallskip, \medskip,
            %% \bigskip, \vpsace etc to make ajustments.
        % ----- LANGUAGES -----
            
        % ----- REFERENCES -----
        \cvsection{References}
            \cvref{Prof.\ Alpha Beta}{Institute}{a.beta@university.edu}
            \divider

            \cvref{Boss\ Gamma Delta}{Business}{g.delta@business.com}
        % ----- REFERENCES -----
        
        % ----- MOST PROUD -----
        % \cvsection{Most Proud of}
        
        % \cvachievement{\faTrophy}{Fantastic Achievement}{and some details about it}\\
        % \divider
        % \cvachievement{\faHeartbeat}{Another achievement}{more details about it of course}\\
        % \divider
        % \cvachievement{\faHeartbeat}{Another achievement}{more details about it of course}
        % ----- MOST PROUD -----
        
        % \cvsection{A Day of My Life}
        
        % Adapted from @Jake's answer from http://tex.stackexchange.com/a/82729/226
        % \wheelchart{outer radius}{inner radius}{
        % comma-separated list of value/text width/color/detail}
        % \wheelchart{1.5cm}{0.5cm}{%
        %   6/8em/accent!30/{Sleep,\\beautiful sleep},
        %   3/8em/accent!40/Hopeful novelist by night,
        %   8/8em/accent!60/Daytime job,
        %   2/10em/accent/Sports and relaxation,
        %   5/6em/accent!20/Spending time with family
        % }
        
        % use ONLY \newpage if you want to force a page break for
        % ONLY the current column
        \newpage
        
        %% Switch to the right column. This will now automatically move to the second
        %% page if the content is too long.
        \switchcolumn
        
        % ----- ABOUT ME -----
        \cvsection{About Me}
            \begin{quote}
                Lorem ipsum dolor sit amet, consectetur adipiscing elit, sed do eiusmod tempor incididunt ut labore et dolore magna aliqua.
            \end{quote}
        % ----- ABOUT ME -----
        
        % ----- EXPERIENCE -----
        \cvsection{Experience}
            \cvevent{Charge}{Company}{Mm YYYY -- Mm YYYY}{City, Country}
            \begin{itemize}
                \item First achievement
                \item Second achievement
                \item Third achievement
            \end{itemize}
            \divider
            
            \cvevent{Charge}{Company}{Mm YYYY -- Mm YYYY}{City, Country}
            \begin{itemize}
                \item First achievement
                \item Second achievement
                \item Third achievement
            \end{itemize}
        % ----- EXPERIENCE -----
        
        % ----- EDUCATION -----
        \cvsection{Education}
            \cvevent{Title}{Institution}{Mm YYYY -- Mm YYYY}{City, Country}
            \begin{itemize}
                \item GPA: 1,23
            \end{itemize}
            \divider
            
            \cvevent{Title}{Institution}{Mm YYYY -- Mm YYYY}{City, Country}
            \begin{itemize}
                \item GPA: 1,23
            \end{itemize}
        % ----- EDUCATION -----
        
        % ----- PROJECTS -----
        \cvsection{Projects}
            \cvevent{Project 1 }{\cvreference{\faGithub}{https://github.com/user/repo}\cvreference{| \faGlobe}{https://project-demo.com/}}{Mm YYYY -- Mm YYYY}{}
            \begin{itemize}
                \item Item 1
                \item Item 2
            \end{itemize}
            \divider
            
            \cvevent{Project 2 }{\cvreference{\faGitlab}{https://gitlab.com/user/repo}\cvreference{| \faGlobe}{https://project-demo.com/}}{Mm YYYY -- Mm YYYY}{}
            \begin{itemize}
                \item Item 1
                \item Item 2
            \end{itemize}
        % ----- PROJECTS -----
    \end{paracol}
\end{document}